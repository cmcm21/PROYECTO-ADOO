\begin{UseCase}{CU8}{Consultar Ventas del Día.}{
		Se requiere una manera sencilla para poder visualizar las ganancias adquiridas y los productos al término del día.
	}
		\UCitem{Versión}{\color{Gray}0.1}
		\UCitem{Autor}{\color{Gray}Vázquez Cruz Fernando Darwin }
		\UCitem{Supervisa}{\color{Gray}Enrique Aguilera}
		\UCitem{Actor}{\hyperlink{Alumno}{Supervisor}}
		\UCitem{Propósito}{Conocer de manera sencilla las ganacias obtenidas durante el día y los productos vendidos para corroborar que las entradas de dinero corresponden con las salidas de productos.}
		\UCitem{Entradas}{Ingresos y productos vendidos por caja}
		\UCitem{Origen}{Teclado, mouse}
		\UCitem{Salidas}{Gráfica con ingresos y gráfica con productos vendidos.)}
		\UCitem{Destino}{Pantalla}
		\UCitem{Precondiciones}{Las cajas deben estar cerradas.}
		\UCitem{Postcondiciones}{Genera gráficas sobre ingresos y salidas de productos.}
		\UCitem{Errores}{Que no se realice una conexión a la base de datos,que el servidor se caiga,que no haya ventas en el día.}
		\UCitem{viene de...}{\UCref{CU0}{Control de acceso}}
		\UCitem{Observaciones}{}
		\UCitem{Estado}{En revisión}
	\end{UseCase}
%--------------------------------------
	\begin{UCtrayectoria}{Principal}
		\UCpaso [\UCactor] Presiona el botón \IUbutton{Ventas} que esta en la pantalla principal \IUref{IU1}{Pantalla principal}
		\UCpaso Verifica que los permisos de usuario sean permisos de supervisor. \Trayref{A}
		\UCpaso[\UCactor] Ve los tipos de reporte disponibles y elige la opción "Reporte de ventas al día".
		\UCpaso Verifica que las cajas estén cerradas. \Trayref{B}
		\UCpaso \UCpaso Verifica las ventas de cada caja y suma los totales por cada caja.
		\UCpaso \UCpaso Verifica los productos vendidos y los agrupa por nombre.
		\UCpaso \UCpaso Genera una gráfica con el total de ingresos del día por caja y en la parte de abajo genera el total de ingresos de la sucursal.
		\UCpaso \UCpaso Genera una gráfica con los productos vendidos y su cantidad.
	\end{UCtrayectoria}

	%--------------------------------------		
	\begin{UCtrayectoriaA}{A}{Permiso Denegado}
			\UCpaso Muestra el Mensaje {\bf MSG4-}`"Cancelado[{\em Permiso Denegado }].".
			\UCpaso Muestra la pantalla \IUref{IU1}{Pantalla Principal}.
		\end{UCtrayectoriaA}
	%--------------------------------------		
		\begin{UCtrayectoriaA}{B}{El Empleado no puede ser agregado}
			\UCpaso Despliega el mensaje \bf {+ MSG1 }.
			\UCpaso[\UCactor] Da clic en el \IUbutton {+ Ok }.
			\UCpaso Continua en el paso 7 del \UCref{CU8}.
		\end{UCtrayectoriaA}

	%--------------------------------------