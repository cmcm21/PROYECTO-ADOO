\begin{UseCase}{CU2}{Consultar Medicamento}{
		
	}
		\UCitem{Versión}{\color{Gray}0.1}
		\UCitem{Autor}{\color{Gray}Felipe Zamora Gachuz}
		\UCitem{Supervisa}{\color{Gray}-}
		\UCitem{Actor}{Cajero}
		\UCitem{Propósito}{Poder saber datos del medicamento}
		\UCitem{Entradas}{Nombre del medicamento, ingrediente activo, }
		\UCitem{Origen}{Teclado}
		\UCitem{Salidas}{Medicamento que coincida con la busqueda }
		\UCitem{Destino}{Pantalla}
		\UCitem{Precondiciones}{Tener medicamentos registrados}
		\UCitem{Postcondiciones}{}
		\UCitem{Errores}{}
		\UCitem{Observaciones}{}
		\UCitem{Estado}{En revision}
	\end{UseCase}
%--------------------------------------
	\begin{UCtrayectoria}{Principal}
		\UCpaso [\UCactor] Tiene que estar en las pantallas de cajero.
		\UCpaso El sistema genera y despliega la pantalla \IUref(IUx){ConsultaMedicamento} \Trayref{A}\Trayref{C}
		\UCpaso [\UCactor] Ingresa el nombre del medicamento o ingrediente activo en la barra de busqueda de \IUref(IUx){ConsultaMedicamento}		
		\UCpaso El sistema filtra los medicamentos segun la coincidencia(se muestra y actualiza la pantalla \IUref(IUx){ConsultaMedicamento}) \Trayref{B}\Trayref{C}
		\UCpaso [\UCactor] Da en el boton aceptar
		\UCpaso El sistema regresa a las pantallas del cajero.
	\end{UCtrayectoria}

%--------------------------------------		
	\begin{UCtrayectoriaA}{A}{Error relacionado con la base de datos}
			\UCpaso Muestra el Mensaje {\bf MSG1-``Error en la Operación''}.
		    \UCpaso El sistema regresa a las pantallas del cajero.
		\end{UCtrayectoriaA}
%----------------------------------------
		\begin{UCtrayectoriaA}{B}{No hay coincidencia}
			\UCpaso Semuestra mensaje no documentado de ``Sin coincidencias''
			\UCpaso Continua en el paso 4 del \UCref{CU2}.
		\end{UCtrayectoriaA}		
%--------------------------------------
		\begin{UCtrayectoriaA}{C}{Da en el boton aceptar}
		\UCpaso El sistema regresa a las pantallas del cajero.
		\end{UCtrayectoriaA}
%%--------------------------------------
%	\begin{UCtrayectoriaA}{D}{Correo Electronico de recuperación de contraseña no se envio}
%			\UCpaso Muestra el Mensaje {\bf MSG04-}``Error [{\em Correo no enviado}] Revisa tu conexión y vuelve a enviar el mensaje .''.
%			\UCpaso[\UCactor] Oprime el botón \IUbutton{Aceptar}
%			\UCpaso Continua en el paso 7 del \UCref{CU2}.
%		\end{UCtrayectoriaA}
%%--------------------------------------
