 \begin{UseCase}{CU5}{Registrar Venta Al Mostrador}{
	EL crecimiento y funcionamiento adecuado de las farmacias de la franquicia se
	basa en las ventas que se realizan en estas, por eso es que el sistema que se solicita, necesita una manera de administrar esas ventas realizadas en cada sucursal.
	}
		\UCitem{Versión}{\color{Gray}0.1}
		\UCitem{Autor}{\color{Gray}Correa Medina Carlos Miguel}
		\UCitem{Supervisa}{\color{Gray}.Darwin}
		\UCitem{Actor}{Cajero}
		\UCitem{Propósito}{Tener un control de las salidas de medicamento por ventas y poder tener un reporte de ventas diarias, al igual que un control de los medicamentos que se venden.}
		\UCitem{Entradas}{cantidad total a pagar, medicamentos que se venden y precio de venta de cada uno,por el momento solo se hacen pagos en efectivo,son Todos los datos requeridos en el formulario \IUref{IU19}{Formulario Venta}}
		\UCitem{Origen}{Teclado}
		\UCitem{Salidas}{comprobante de venta, medicamento(s)}
		\UCitem{Destino}{Pantalla,impresora}
		\UCitem{Precondiciones}{Los medicamentos a vender deben estar registrados en el sistema, el supervisor debió haber abierto la caja con anterioridad \UCref{CU9}{Abrir Caja}.}
		\UCitem{Postcondiciones}{Disminuye la cantidad de los medicamentos que se vendieron, se tiene una venta más registrada en el sistema.}
		\UCitem{Errores}{Que no se cuente con conexión a Internet,no haya energía eléctrica para utilizar un computador,Que no se realice una conexión a la base de datos,que el servidor se caiga,que los datos proporcionados estén erróneos, que el o los medicamentos estén registrados en el sistema pero no estén disponibles en la sucursal.}
		\UCitem{viene de...}{\UCref{CU0}{Control de acceso}}
		\UCitem{Observaciones}{}
		\UCitem{Estado}{En revisión}
	\end{UseCase}
%--------------------------------------
	\begin{UCtrayectoria}{Principal}
		\UCpaso incluye al caso de uso \UCref{IU0}{Control de Acceso}.
		\UCpaso [\UCactor] presiona el botón \IUbutton{Ventas} que esta en la pantalla principal \IUref{IU1}{Pantalla principal}
		\UCpaso verifica que los permisos de usuario sean permisos de cajero. \Trayref{A}
		\UCpaso Despliega el Formulario \IUref{IU19}{Formulario Ventas} 
		\UCpaso busca el id del cajero que esta actualmente con la sesión activa y llena el campo de ``Empleado" con ese id.
		\UCpaso [\UCactor] en el primer campo de formulario ``Cliente" Selecciona la opción ``Publico en general".
		\UCpaso identifica la opción seleccionada por el [\UCactor] y siendo esta opción ``Publico general"" deja el formulario como esta.
		\UCpaso [\UCactor] lee el código de barras del medicamento con el lector de código de barras
		\UCpaso Llena el campo de ``Medicamentos" con el código de barras que se introdujo del lector.\Trayref{B}.
		\UCpaso Recibe un ``Enter" del lector de código de barras y añade otra fila para agregar un nuevo medicamento en el campo ``Medicamentos".
		\UCpaso Agrega en una lista que crea temporalmente, el nombre del  medicamento y el precio de venta del medicamento que fue leído por el lector de código de barras.
		\UCpaso muestra la lista temporal de los medicamentos en el campo ``Desglose de medicamentos y precios".\Trayref{C}
		\UCpaso  Calcula el Total de los precios de venta de los medicamentos en la lista temporal y muestra la cantidad en el campo ``Valor Total".
		\UCpaso [\UCactor] Presiona el botón \IUbutton{Guardar}.
		\UCpaso Muestra el Mensaje {\bf MSG6-}``Confirmar[{\em Confirmar Operación }]¿Los datos del formulario son correctos?.''.
		\UCpaso Presiona el botón \IUbutton{Aceptar}
		\UCpaso Verifica que ningún campo del formulario este vació \Trayref{D}
		\UCpaso Muestra el Mensaje {\bf MSG0-}``Exito[{\em Operación Realizada con exito }].''.
		\UCpaso Muestra la pantalla \IUref{IU1}{Pantalla Principal}
	\end{UCtrayectoria}

%--------------------------------------		
	\begin{UCtrayectoriaA}{A}{Permiso Denegado}
			\UCpaso Muestra el Mensaje {\bf MSG4-}``Cancelado[{\em Permiso Denegado }].''.
			\UCpaso Regresa al paso 1 de este caso de uso.
		\end{UCtrayectoriaA}

%--------------------------------------
		\begin{UCtrayectoriaA}{B}{El lector de barras no pudo leer bien el código}
			\UCpaso sigue su ejecución sin llenar el campo ``Medicamentos""
			\UCpaso Continua en el paso 8 de este caso de uso.	
		\end{UCtrayectoriaA}
%----------------------------------------
		\begin{UCtrayectoriaA}{C}{Lista de medicamentos muy larga }
			\UCpaso Muestra en el formulario \IUref{IU19}{Formulario Ventas} un 					``scroll" en el campo de ``Desglose de medicamentos y precios"
			\UCpaso Regresa al paso 13 de este caso de uso.
		\end{UCtrayectoriaA}		
%--------------------------------------
			\begin{UCtrayectoriaA}{D}{Existe por lo menos un campo obligatoria que esta vació}
			\UCpaso remarca con Rojo los campos obligatorios que están vacíos 
			y pone la leyenda ``Campo Obligatorio".
			\UCpaso Muestra el Mensaje {\bf MSG1-}``Error en la operación [{\em Campos obligatorios vacíos}] Los campos llenados con Rojo no pueden estar vacíos.''.
			\UCpaso[\UCactor] Oprime el botón \IUbutton{Aceptar}
			\UCpaso Regresa a la pantalla \IUref{IU19}{Formulario Ventas} Mostrando los campos en rojo y dejando la información introducida por el [\UCactor] en los pasos anteriores de este casos de uso
		\end{UCtrayectoriaA}
