 \begin{UseCase}{CU1}{Registro de medicamento}{
 		El Dueño es el unico actor que puede hacer un registro de un medicamento en el sistema, despliegamos la pantalla  \IUref{IU5}{FormularioMedicamento}, verificamos proveedores y medicamentos, actualizamos inventario, guardamos en el sistema.
 	}
 	\UCitem{Versión}{\color{Gray}0.1}
 	\UCitem{Autor}{\color{Gray}Felipe Zamora Gachuz}
 	\UCitem{Supervisa}{\color{Gray}.Miguel Correa Medina}
 	\UCitem{Actor}{Dueño}
 	\UCitem{Propósito}{El dueño registra un nuevo medicamento en el sistema para poder hacer las ventas posteriores}
 	\UCitem{Entradas}{ }
 	\UCitem{Origen}{Teclado, Base de datos, lector de código de barras}
 	\UCitem{Salidas}{Nombre de medicamento,existencias, dosis, lote, advertencias, fecha de caducidad del lote,  código de barras, presentación, via de administración, precio de venta, precio de compra, ingredientes activos, cantidad de ingrediente activo, nombre del proveedor, RCF del proveedor, estado del medicamento}
 	\UCitem{Destino}{Pantalla \IUref{IU17}{FormularioMedicamento}}
 	\UCitem{Precondiciones}{No tener ya registrado el medicamento,tener registro del proveedor, tener permisos de dueño, tener existencias del medicamento a agregar. }
 	\UCitem{Postcondiciones}{Modifica el inventario, agrega un registro de nuevo medicamento en la patalla \IUref{IU17}{TablaMedicamentos}}
 	\UCitem{Errores}{Medicamento ya registrado, proveedor no encontrado, error de conexión}
 	\UCitem{Observaciones}{}
 	\UCitem{Estado}{En revision}
 \end{UseCase}
 %--------------------------------------
 \begin{UCtrayectoria}{Principal}
 	\UCpaso Incluye el caso de uso \UCref{CU0}{Control de acceso}
 	\UCpaso [\UCactor] Tiene que tener permisos de dueño.
 	\UCpaso [\UCactor] Abre la pantalla \IUref{IU17}{TablaMedicamento} en la pantalla principal.
 	\UCpaso [\UCactor]Presiona el boton \IUbutton{+Nuevo} de la pantalla \IUref{IU17}{TablaMedicamento}.
 	\UCpaso El sistema genera la lista despliegable de los proveedores \Trayref{A}.
 	\UCpaso El sistema genera la lista despliegable de presentación(tabletas, capsulas ...) \Trayref{A}.
 	\UCpaso El sistema genera la lista despliegable de via de administración \Trayref{A}.%error en la lista
 	\UCpaso El sistema genera la lista despliegable de los laboratorios \Trayref{A}.
 	\UCpaso El sistema genera la lista despliegable de ingredientes activos \Trayref{A}.%error en la lista
 	\UCpaso El sistema despliega la pantalla \IUref{IU5}{FormularioMedicamento} con las listas del paso anterior.
 	\UCpaso [\UCactor] Tiene que llenar los datos solicitados por el formulario del medicamento y proveedor.\Trayref{H}%cancelar vv
 	\UCpaso	[\UCactor]Preciona el boton \IUbutton{Guardar}.\Trayref{H}%cancelar
 	%----------------
 	\UCpaso El sistema verifica el formulario no este vacio \Trayref{B}.
 	\UCpaso	El sistema verifica el nombre del medicamento \Trayref{C}
 	\UCpaso	El sistema verifica que no hay duplicado en el código de barras \Trayref{D}.
 	\UCpaso El sistema verifica el formato de la  fecha de caducidad \Trayref{E}.
 	\UCpaso El sistema verifica que la fecha de caducida no sea próxima a menor de 2 años\Trayref{F}.%%%%
 	\UCpaso El sistema verifica que le precio de venta sea mayor al de compra \Trayref{G}.
 	%----------------
 	%verificar alfabetico,  numerico, correo, vacio(objeto medicamento) 
 	%%%%%
 	\UCpaso Despliega el mensaje {\bf MSG06-} {Confirmar Operación}. 
 	\UCpaso [\UCactor] Presiona el botón \IUbutton{Si!} \Trayref{H}.
 	\UCpaso Guarda la información del medicamento.
 	%%%%% 
 	%%%%%
 	\UCpaso Despliega el mensaje {\bf MSG0-} {Operación Exitosa}
 	\UCpaso [\UCactor] Presiona el botón \IUbutton{OK}.
 	\UCpaso El sistema regresa a la pantalla \IUref{IU17}{TablaMedicamento} .
 	
 \end{UCtrayectoria}
 
 %--------------------------------------		
 \begin{UCtrayectoriaA}{A}{No puede generar lista despliegable}
 	\UCpaso El sistema muestra la pantalla {\bf MSG1-``Error en la Operación''}
 	\UCpaso EL sistema muestra la pantalla \IUref{IU17}{FormularioMedicamento}.
 \end{UCtrayectoriaA}
 %----------------------------------------
 \begin{UCtrayectoriaA}{B}{Hay datos del formulario que estan vacios}
 	\UCpaso El sistema muestra abajo en la pantalla mensaje existen campos vacios.
 	\UCpaso Continua en el paso 7 del \UCref{CU1}.
 \end{UCtrayectoriaA}		
 %--------------------------------------
 \begin{UCtrayectoriaA}{C}{El nombre del medicamento tiene un carácter no alfabetico}
 	\UCpaso El sistema muestra abajo en la pantalla mensaje de error nombre no valido.
 	\UCpaso Continua en el paso 7 del \UCref{CU1}.
 \end{UCtrayectoriaA}
 %--------------------------------------
 %--------------------------------------
 \begin{UCtrayectoriaA}{D}{Duplicado codigo de barras}
 	\UCpaso El sistema muestra abajo en la pantalla mensaje de duplicado del código de barras.
 	\UCpaso Continua en el paso 7 del \UCref{CU1}.
 \end{UCtrayectoriaA}
 %--------------------------------------
 \begin{UCtrayectoriaA}{E}{Formato de la fecha de caducidad invalido}
 	\UCpaso El sistema muestra abajo en la pantalla mensaje de error el formato de la fecha de caducidad.
 	\UCpaso Continua en el paso 7 del \UCref{CU1}.
 \end{UCtrayectoriaA}
 %--------------------------------------
 \begin{UCtrayectoriaA}{F}{La fecha de caducidad es menor a 2 años}
 	\UCpaso El sistema muestra abajo en la pantalla mensaje de error fecha de caducidad fuera del rango permitido de 2 años.
 	\UCpaso Continua en el paso 7 del \UCref{CU1}.
 \end{UCtrayectoriaA}
 %--------------------------------------
 \begin{UCtrayectoriaA}{G}{Condición de precio es falsa}
 	\UCpaso El sistema muestra abajo en la pantalla mensaje de error en los precios de compra y venta.
 	\UCpaso Continua en el paso 7 del \UCref{CU1}.
 \end{UCtrayectoriaA}
 %--------------------------------------
 \begin{UCtrayectoriaA}{H}{Preciona el boton cancelar}
 	\UCpaso El sistema regresa a la pantalla  \IUref{IU17}{FormularioMedicamento}.
 \end{UCtrayectoriaA}		