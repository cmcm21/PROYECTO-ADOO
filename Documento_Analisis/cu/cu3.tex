\begin{UseCase}{CU3}{Registrar cajero}{
		Registramos un nuevo cajero en el sistema, validamos los cajeros por sucursal asi como datos del cajero. 	
	}
	\UCitem{Versión}{\color{Gray}0.1.5}
	\UCitem{Autor}{\color{Gray}Felipe Zamora Gachuz}
	\UCitem{Supervisa}{\color{Gray}Miguel Correa Medina}
	\UCitem{Actor}{Dueño}
	\UCitem{Propósito}{}%Registrar cajeros en la sucursales donde faltan cajeros}
	\UCitem{Entradas}{Correo del cajero,nombre, apellidos del cajero, telefono, salario, sucursal, turno}
	\UCitem{Origen}{Teclado}
	\UCitem{Salidas}{Registro de un cajero en una unica sucursal}
	\UCitem{Destino}{Pantalla}
	\UCitem{Precondiciones}{No tener dado de alta al cajero en el sistema en general, la sucursal debe tener un  numero valido de cajeros en el turno por asignar}
	\UCitem{Postcondiciones}{Registro en el sistema de un cajero nuevo en una sucursal}
	\UCitem{Errores}{No hay respuesta del servidor, no hay conexión a internet, el correo ya este registrado en el sistema}
	\UCitem{Observaciones}{}
	\UCitem{Estado}{En revisión}
\end{UseCase}
%--------------------------------------
\begin{UCtrayectoria}{Principal}
	\UCpaso Incluye el caso de uso \UCref{CU0}{Control de acceso}.
	\UCpaso [\UCactor] Tiene que tener permisos de dueño.
	\UCpaso [\UCactor] Presiona la opción cajero en la pantalla principal.
	\UCpaso El sistema genera la lista despliegable de las sucursales \Trayref{A}.
	\UCpaso El sistema despliega la pantalla \IUref{IU18}{TablaCajeros}.
	\UCpaso [\UCactor]Preciona el boton  \IUbutton{+Nuevo} de la pantalla \IUref{IU18}{TablaCajeros}
	\UCpaso El sistema genera la lista despliegable de las sucursales disponibles para agregar cajero\Trayref{A}.
	\UCpaso El sistema despliega la pantalla \IUref{IU7}{FormularioCajero}\Trayref{H}.
	\UCpaso [\UCactor] Introduce los datos del formulario \IUref{IU7}{FormularioCajero} \Trayref{H}.
	\UCpaso [\UCactor] Selecciona la sucursal en la que operara el cajero\Trayref{H}.
	\UCpaso El sistema genera la lista despliegable de los  turnos disponibles de la sucursal seleccionada \Trayref{A}.
	\UCpaso [\UCactor] Selecciona el turno del cajero en que se registra\Trayref{H}.
	\UCpaso [\UCactor]Presiona el boton \IUbutton{guardar} \Trayref{H}.
	
	%---------------
	%---------------
	\UCpaso El sistema verifica el formulario no este vacio \Trayref{B}.
	\UCpaso	El sistema verifica el correo es valido \Trayref{C}.
	%\UCpaso El sistema verifica el correo tenga arroba \Trayref{D}.
	%\UCpaso El sistema verifica el correo termine en .com o .mx \Trayref{E}
	\UCpaso El sistema verifica que salario sea numerico \Trayref{F}.%numeros en salario
	%%
	\UCpaso Despliega el mensaje {\bf MSG06-} {Confirmar Operación}. 
	\UCpaso [\UCactor] Presiona el botón \IUbutton{Si!} \Trayref{H}.
	\UCpaso Guarda la información del cajero.
	%%
	\UCpaso Despliega el mensaje {\bf MSG0-} {Operación Exitosa}
	\UCpaso [\UCactor] Presiona el botón \IUbutton{OK}.
	
	%			
	%		\UCpaso EL sistema registra el nuevo cajero en el sistema
	%		\UCpaso El sistema regresa a la pantalla\IUref{IU18}{TablaCajeros}.
	
	
\end{UCtrayectoria}

%\textbf{NOTA:El numero de cajeros y supervisores es de 3 por sucursal.}
%-------------------------------------------------------------------------

\begin{UCtrayectoriaA}{A}{Sucursal no encontrada}
	\UCpaso El sistema muestra la pantalla {\bf MSG1-``Error en la Operación''}.
	\UCpaso El sistema regresa a la \IUref{IU18}{TablaCajeros}.
\end{UCtrayectoriaA}
%-------------------------------------------------------------------------

\begin{UCtrayectoriaA}{B}{Hay datos del formulario que estan vacios}
	\UCpaso El sistema muestra abajo en la pantalla el/los nombres de los campos vacios.
	\UCpaso Continua en el paso 9 del \UCref{CU3}.
\end{UCtrayectoriaA}	
%-------------------------------------------------------------------------

\begin{UCtrayectoriaA}{C}{Error no se detecto formato valido}
	\UCpaso  El sistema muestra la pantalla {\bf MSG1-``Error en la Operación''}.
	\UCpaso El sistema limpia el campo de correo.
	\UCpaso Continua en el paso 9 del \UCref{CU3}.
\end{UCtrayectoriaA}
%-------------------------------------------------------------------------
%
%\begin{UCtrayectoriaA}{D}{No se detecto arroba}
%	\UCpaso  El sistema muestra la pantalla {\bf MSG1-``Error en la Operación''}.
%	\UCpaso El sistema limpia el campo de correo.
%	\UCpaso Continua en el paso 9 del \UCref{CU3}.
%\end{UCtrayectoriaA}
%
%%-------------------------------------------------------------------------
%
%\begin{UCtrayectoriaA}{E}{No se deteto terminacion .com o .mx}
%	\UCpaso  El sistema muestra la pantalla {\bf MSG1-``Error en la Operación''}.
%	\UCpaso El sistema limpia el campo de correo.
%	\UCpaso Continua en el paso 9 del \UCref{CU3}.
%\end{UCtrayectoriaA}

%-------------------------------------------------------------------------

\begin{UCtrayectoriaA}{F}{Datos alfanumericos detectados}
	\UCpaso  El sistema muestra la pantalla {\bf MSG1-``Error en la Operación''}.
	\UCpaso El sistema limpia el campo de salario.
	\UCpaso Continua en el paso 9 del \UCref{CU3}.
\end{UCtrayectoriaA}

%--------------------------------------
\begin{UCtrayectoriaA}{H}{Preciona el boton cancelar}
	\UCpaso El sistema regresa a la pantalla  \IUref{IU18}{TablaCajeros}.
\end{UCtrayectoriaA}